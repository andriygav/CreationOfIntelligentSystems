\documentclass[10pt,pdf,hyperref={unicode}]{beamer}

\mode<presentation>
{
\usetheme{boxes}
\beamertemplatenavigationsymbolsempty

\setbeamertemplate{footline}[page number]
\setbeamersize{text margin left=0.5em, text margin right=0.5em}
}

\usepackage[utf8]{inputenc}
\usepackage[english, russian]{babel}
\usepackage{bm}
\usepackage{multirow}
\usepackage{ragged2e}
\usepackage{indentfirst}
\usepackage{multicol}
\usepackage{subfig}
\usepackage{amsmath,amssymb}
\usepackage{enumerate}
\usepackage{mathtools}
\usepackage{comment}
\usepackage{multicol}

\usepackage[all]{xy}

\usepackage{tikz}
\usetikzlibrary{positioning,arrows}

\tikzstyle{name} = [parameters]
\definecolor{name}{rgb}{0.5,0.5,0.5}

\usepackage{caption}
\captionsetup{skip=0pt,belowskip=0pt}

\newtheorem{rustheorem}{Теорема}
\newtheorem{russtatement}{Утверждение}
\newtheorem{rusdefinition}{Определение}

% colors
\definecolor{darkgreen}{rgb}{0.0, 0.2, 0.13}
\definecolor{darkcyan}{rgb}{0.0, 0.55, 0.55}

\AtBeginEnvironment{figure}{\setcounter{subfigure}{0}}

\captionsetup[subfloat]{labelformat=empty}

%----------------------------------------------------------------------------------------------------------

\title[Вводная лекция]{Создание Интеллектуальных Систем \\ Выбор задач}
\author{А.\,В.\,Грабовой}

\institute[]{Московский физико-технический институт}
% \date[2022]{\small 10\;февраля\;2022\,г.}

%---------------------------------------------------------------------------------------------------------
\begin{document}

\begin{frame}
\titlepage
\end{frame}

%----------------------------------------------------------------------------------------------------------
\section{Задачи для команд}
\begin{frame}{Задачи для команд}
\bigskip

Рассматривается задача аппроксимации показаний fMRI и iEEG для разных типов стимулов:
\begin{enumerate}
    \item аппроксимация fMRI по данным видео потока;
    \item аппроксимация iEEG по данным видео потока;
    \item аппроксимация fMRI по данным звукового потока;
    \item аппроксимация iEEG по данным звукового потока.
\end{enumerate}

\bigskip

Выборка для аппроксимации представлена в работе J. Berezutskay\footnotemark, в которой присутствуют различные типы параллельных сигналов.

\bigskip
\footnotetext[1]{\textit{Berezutskaya J., et al} Open multimodal iEEG-fMRI dataset from naturalistic stimulation with a short audiovisual film // Sci Data 9, 91, 2022.}
\end{frame}


\end{document}