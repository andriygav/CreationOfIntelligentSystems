\documentclass[10pt,pdf,hyperref={unicode}]{beamer}

\mode<presentation>
{
\usetheme{boxes}
\beamertemplatenavigationsymbolsempty

\setbeamertemplate{footline}[page number]
\setbeamersize{text margin left=0.5em, text margin right=0.5em}
}

\usepackage[utf8]{inputenc}
\usepackage[english, russian]{babel}
\usepackage{bm}
\usepackage{multirow}
\usepackage{ragged2e}
\usepackage{indentfirst}
\usepackage{multicol}
\usepackage{subfig}
\usepackage{amsmath,amssymb}
\usepackage{enumerate}
\usepackage{mathtools}
\usepackage{comment}
\usepackage{multicol}

\usepackage[all]{xy}

\usepackage{tikz}
\usetikzlibrary{positioning,arrows}

\tikzstyle{name} = [parameters]
\definecolor{name}{rgb}{0.5,0.5,0.5}

\usepackage{caption}
\captionsetup{skip=0pt,belowskip=0pt}

\newtheorem{rustheorem}{Теорема}
\newtheorem{russtatement}{Утверждение}
\newtheorem{rusdefinition}{Определение}

% colors
\definecolor{darkgreen}{rgb}{0.0, 0.2, 0.13}
\definecolor{darkcyan}{rgb}{0.0, 0.55, 0.55}

\AtBeginEnvironment{figure}{\setcounter{subfigure}{0}}

\captionsetup[subfloat]{labelformat=empty}

%----------------------------------------------------------------------------------------------------------

\title[Вводная лекция]{Создание Интеллектуальных Систем \\ Вводная лекция}
\author{А.\,В.\,Грабовой}

\institute[]{Московский физико-технический институт}
% \date[2022]{\small 10\;февраля\;2022\,г.}

%---------------------------------------------------------------------------------------------------------
\begin{document}

\begin{frame}
\titlepage
\end{frame}

%----------------------------------------------------------------------------------------------------------
\section{Идея курса}
\begin{frame}{Идея курса}
\bigskip
\begin{enumerate}[1)]
\justifying
\item расширение курса ``Моя первая научная статья'',
\item важна работа в командах --- 3--4 человека,
\item еженедельная презентация работы команды за неделю:
    \begin{itemize}
    \justifying
        \item каждые 3 недели выбирается один ответственный человек,
        \item каждую неделю ответственный готовит презентацию на 5 минут о результатах всей команды,
        \item в презентацию входит краткий план работ на следующую неделю.
    \end{itemize}
\item система чекпоинтов:
    \begin{itemize}
    \justifying
        \item проект условно делиться на 4 части,
        \item каждая часть рассчитана на 3 недели,
        \item каждая часть проверяется по истечению срока соотвествующей части.
    \end{itemize}
\end{enumerate}

\footnotetext[1]{Презентация должна содержать 2-3 слайда}
\footnotetext[2]{На подготовку презентации не более 15 минут}
\footnotetext[3]{Ответственный человек меняется каждые 3 недели}
\end{frame}

%----------------------------------------------------------------------------------------------------------
\section{План отчетности по курсу}
\begin{frame}{План отчетности по курсу}
\bigskip
\begin{table}[]
\begin{tabular}{|c|c|c|}
\hline
Дата  & Чекпоинты                              & Балл \\ \hline
13.09 & -                                      &      \\ \hline
13.09 & Раздача проектов                       & 0    \\ \hline
20.09 & -                                      &      \\ \hline
27.09 & -                                      &      \\ \hline
04.10 & Первый чекпоинт (анализ работ)         & 3    \\ \hline
11.10 & -                                      &      \\ \hline
18.10 & -                                      &      \\ \hline
25.10 & Второй чекпоинт (теоретическая часть)  & 3    \\ \hline
01.11 & -                                      &      \\ \hline
08.11 & -                                      &      \\ \hline
15.11 & Третий чекпоинт (эксперимент)          & 3    \\ \hline
22.11 & -                                      &      \\ \hline
29.11 & -                                      &      \\ \hline
06.12 & Четвертый чекпоинт (структура проекта) & 3    \\ \hline
13.12 & Зачет                                  & 0    \\ \hline
\end{tabular}
\end{table}
\end{frame}

%----------------------------------------------------------------------------------------------------------
\section{План лекций}
\begin{frame}{План лекций}
\bigskip
\begin{table}[]
\begin{tabular}{|c|l|}
\hline
\textbf{Дата} & \multicolumn{1}{c|}{\textbf{Тема лекции}}                 \\ \hline
13.09         & Вводная лекция                                            \\ \hline
13.09         & Обсуждения и выбор проектов                               \\ \hline
20.09         & Краткий экскурс по LaTeX, GitHub, Jupyter Notebook        \\ \hline
27.09         & Структура научной статьи                                  \\ \hline
04.10         & Первый чекпоинт (анализ работ)                            \\ \hline
11.10         & Структура научного проекта                                \\ \hline
18.10         & Разбор основных проблем при работе с Jupyter Notebook     \\ \hline
25.10         & Второй чекпоинт (теоретическая часть)                     \\ \hline
01.11         & Повторяемый вычислительный эксперимент с внешними данными \\ \hline
08.11         & Написания повторяемого кода вычислительного эксперимента  \\ \hline
15.11         & Третий чекпоинт (эксперимент)                             \\ \hline
22.11         & Документация кода вычислительного эксперимента            \\ \hline
29.11         & Докернизация кода вычислительного эксперимента            \\ \hline
06.12         & Четвертый чекпоинт (структура проекта)                    \\ \hline
13.12         & Зачет                                                     \\ \hline
\end{tabular}
\end{table}
\end{frame}

%----------------------------------------------------------------------------------------------------------
\section{Игра в команду}
\begin{frame}{Игра в команду}
\bigskip

\begin{itemize}
    \item в проекте участвует 3--4 человека,
    \item перед каждым этапом проекта назначается один ответственный человек за данный этап;
    \item ответственный распределяет задачи в рамках одного этапа;
    \item ответственный следит за выполнением задач всех участников;
    \item на каждом семинаре ответственный делает краткий доклад по выполненным задачам и по плану на следующую неделю;
    \item при чекпоенте рассказывается доклад по всему этапу.
\end{itemize}
\end{frame}

\end{document}